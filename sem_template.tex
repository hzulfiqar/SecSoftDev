
%% *** Authors should verify (and, if needed, correct) their LaTeX system  ***
%% *** with the testflow diagnostic prior to trusting their LaTeX platform ***
%% *** with production work. IEEE's font choices can trigger bugs that do  ***
%% *** not appear when using other class files.                            ***
%% Testflow can be obtained at:
%% http://www.ctan.org/tex-archive/macros/latex/contrib/supported/IEEEtran/testflow


%
%   -> R E A D M E !!!
%
% You can get information about LaTeX here: 
%      http://www.ctan.org/tex-archive/info/lshort/ 
%
% About the usage of this IEEE-style here:
%      IEEEtran_HOWTO.pdf
%
% and the IEEE BibTex-style here:
%      IEEEtran_bst_HOWTO.pdf
%
% For creating your own bibliography file or adding entries, you can use JabRef.
%      http://jabref.sourceforge.net/
%
% If you have questions how to use the fields in BibTex or which type of 
% reference to use, see the example .bib file. There are many commented BibTex
% entries.
% If you have further questions ask the Internet, then us!



\documentclass[a4paper,conference]{IEEEtran}

% some very useful LaTeX packages include:
\usepackage{cite}       % http://www.ctan.org/tex-archive/macros/latex/contrib/supported/cite/
\ifx\pdfoutput\undefined
\usepackage{graphicx}   % http://www.ctan.org/tex-archive/macros/latex/required/graphics/
\else
\usepackage[pdftex]{graphicx}
\fi
\usepackage{subfigure}  % http://www.ctan.org/tex-archive/macros/latex/contrib/supported/subfigure/
\usepackage{url}        % http://www.ctan.org/tex-archive/macros/latex/contrib/other/misc/
\usepackage{amsmath}    % http://www.ctan.org/tex-archive/macros/latex/required/amslatex/math/
\interdisplaylinepenalty=2500
\ifx\pdfoutput\undefined
\usepackage[hypertex]{hyperref}
\else                   % http://www.ctan.org/tex-archive/macros/latex/contrib/supported/hyperref/
\usepackage[pdftex,hypertexnames=false]{hyperref}
\fi

% correct bad hyphenation here
\hyphenation{op-tical net-works semi-conduc-tor IEEEtran}


\begin{document}

% paper title
\title{Integration of Static Security Code Analysis in Continuous Integration Lifecycles}


% author names and affiliations
\author{\authorblockA{Sebastian Funke, Brian Pfretzschner, Hamza Zulfiqar}
\authorblockA{Center for Advanced Security Research Darmstadt\\
Department of Computer Science\\
Technische Universit\"at Darmstadt, Germany}}


\maketitle


%
% text area
%

\begin{abstract}
The abstract goes here.
\end{abstract}


\section{Introduction}



\section{Dump}

\subsection{CI Server}

\subsubsection{Open Source}
\begin{itemize}
	\item \href{http://jenkins-ci.org}{Jenkins}
	\item \href{http://continuum.apache.org}{Apache Continuum}
\end{itemize}

\subsubsection{Commercial}
\begin{itemize}
	\item \href{http://www.jetbrains.com/teamcity}{TeamCity}\\
	Supports static code analysis. Also, it can import analysis reports produced by these tools: PMD, PMD/CPD, FindBugs, Checkstyle or JSLint.
\end{itemize}

\subsection{Tools}
\begin{itemize}
	\item SonarQube
\end{itemize}

\subsection{Might be interesting to read}
\begin{itemize}
	\item \href{http://www.crosstalkonline.org/storage/issue-archives/2010/201003/201003-Stiehm.pdf}{Building Security In Using Continuous Integration}
\end{itemize}




\section{Now your Stuff}
Your stuff goes here.

\subsection{Subsection}
A reference \cite{Akyildiz2002}.

\subsubsection{Subsubsection}



%An example of a floating figure using the graphicx package.
%Note that \label must occur AFTER (or within) \caption.
%For figures, \caption should occur after the \includegraphics.
%
%\begin{figure}
%\centering
%\includegraphics[width=2.5in]{myfigure}
% where an .pdf, .jpg, .png, .gif,... suffix will be assumed for pdflatex
%\caption{Simulation Results}
%\label{fig_sim}
%\end{figure}


%An example of a double column floating figure using two subfigures.
%(The subfigure.sty package must be loaded for this to work.)
%The subfigure \label commands are set within each subfigure command, the
%\label for the overall fgure must come after \caption.
%\hfil must be used as a separator to get equal spacing
%
%\begin{figure*}
%\centerline{\subfigure[Case I]{\includegraphics[width=2.5in]{subfigcase1}
% where an .pdf, .jpg, .png, .gif,... suffix will be assumed for pdflatex
%\label{fig_first_case}}
%\hfil
%\subfigure[Case II]{\includegraphics[width=2.5in]{subfigcase2}
% where an .pdf, .jpg, .png, .gif,... suffix will be assumed for pdflatex
%\label{fig_second_case}}}
%\caption{Simulation results}
%\label{fig_sim2}
%\end{figure*}



%An example of a floating table. Note that, for IEEE style tables, the 
%\caption command should come BEFORE the table. Table text will default to
%\footnotesize as IEEE normally uses this smaller font for tables.
%The \label must come after \caption as always.
%
%\begin{table}[!bh]
% increase table row spacing, adjust to taste
%\renewcommand{\arraystretch}{1.3}
%\caption{An Example of a Table}
%\label{table_example}
%\begin{center}
%\begin{tabular}{|c||c|}
%\hline
%One & Two\\
%\hline
%Three & Four\\
%\hline
%\end{tabular}
%\end{center}
%\end{table}


\section{Conclusion}
The conclusion goes here.

%
% references section
%

\bibliographystyle{IEEEtranS}
\bibliography{sem_template}     % argument is your BibTeX bibliography database(s)
%\nocite{*}

% that's all folks
\end{document}


