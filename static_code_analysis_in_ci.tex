%%
%% static_code_analysis_in_ci.tex
%% V0.1
%% 2015/01/22
%% by 
%% Sebastian Funke
%% Hamza Zulfiqar
%% Brian Pfretzschner
%% See:
%% https://github.com/hzulfiqar/SecSoftDev
%% for current contact information.
%%


\documentclass[conference]{IEEEtran}


% *** PACKAGES ***
%\usepackage{algorithmic}
%\usepackage{array}
%\usepackage{mdwmath}
%\usepackage{mdwtab}
%\usepackage{eqparbox}
%\usepackage{fixltx2e}
%\usepackage{stfloats}
\usepackage{cite}       % http://www.ctan.org/tex-archive/macros/latex/contrib/supported/cite/
\ifx\pdfoutput\undefined
\usepackage{graphicx}   % http://www.ctan.org/tex-archive/macros/latex/required/graphics/
\else
\usepackage[pdftex]{graphicx}
\fi
\usepackage{subfigure}  % http://www.ctan.org/tex-archive/macros/latex/contrib/supported/subfigure/
\usepackage{url}        % http://www.ctan.org/tex-archive/macros/latex/contrib/other/misc/
\usepackage[cmex10]{amsmath}    % http://www.ctan.org/tex-archive/macros/latex/required/amslatex/math/
%\usepackage{amsfonts}
\interdisplaylinepenalty=2500
\ifx\pdfoutput\undefined
\usepackage[hypertex]{hyperref}
\else                   % http://www.ctan.org/tex-archive/macros/latex/contrib/supported/hyperref/
\usepackage[pdftex,hypertexnames=false]{hyperref}
\fi
\usepackage[colorinlistoftodos,prependcaption,textsize=tiny]{todonotes}
\usepackage{listings}
\lstset{frame=single,captionpos=b}

% correct bad hyphenation here
\hyphenation{op-tical net-works semi-conduc-tor}

\begin{document}
%
% paper title
\title{Integration of Static Security Code Analysis\\in Continuous Integration Lifecycles}



%\author{\IEEEauthorblockN{Sebastian Funke}
%\IEEEauthorblockA{Secure Software Engineering\\
%TU Darmstadt\\
%sebastian.funke@stud.tu-darmstadt.de}
%\and
%\IEEEauthorblockN{Brian Pfretzschner}
%\IEEEauthorblockA{Secure Software Engineering\\
%TU Darmstadt\\
%brian.pfretzschner@stud.tu-darmstadt.de}
%\and
%\IEEEauthorblockN{Hamza Zulfiqar}
%\IEEEauthorblockA{Secure Software Engineering\\
%TU Darmstadt\\
%hamza.zulfiqar@stud.tu-darmstadt.de}}



\author{\authorblockA{Sebastian Funke, Brian Pfretzschner, Hamza Zulfiqar}
	\authorblockA{Center for Advanced Security Research Darmstadt\\
		Department of Computer Science\\
		Technische Universit\"at Darmstadt, Germany}}




% use for special paper notices
%\IEEEspecialpapernotice{(Invited Paper)}




% make the title area
\maketitle


\begin{abstract}
\boldmath
Static code analysis should run frequently in an continuous integration lifecycle. Each run produces lots of information that need to be reviewed, evaluated and integrated in the ongoing development process. Therefore, the analysis results should be reworked, concentrated and presented in a helpful manner.
\end{abstract}

% no keywords

% For peerreview papers, this IEEEtran command inserts a page break and
% creates the second title. It will be ignored for other modes.
\IEEEpeerreviewmaketitle



\section{Introduction}
% no \IEEEPARstart
\todo[inline]{Content from our slides...about input validation and how static code analysis works.}
\cite{SecurityinCI}

\todo{Explain PMD, FindBugs, cppcheck, RATs, maybe FlowDroid}

Building secure software systems gets more and more important. Not just private hackers are attacking our software, also foreign and even western governments put in great effort in breaking into important systems\cite{NSAHacking}.


Therefore we have to improve the software development process in terms of early and continuous security analysis of the source code.
Such static analysis tools can be applied independently during the development process at different stages.
One option would be the integration of analyzers within the IDE of the developer.
Another possibility is the combination of those tools in a continuous integration system, to collect issues during or after the build process.
Besides CI, IDE and the development tool chain can be extended with code quality management systems, as platform for analyze the code and manage the analyzed issues.


In our paper, we compare and evaluate the issue reporting capabilities in two CI tools (Jenkins and Teamcity), two code quality management tools (SonarQube, Codeclimate) and the IDE Eclipse.
We integrated the popular analyzers FindBugs and PMD in each development tool above and executed them on a Java test project (JEdit\footnotetext{http://www.jedit.org/}) with a variety of issues.


We address that what needs validation, how to perform input validation and how to respond when an input fails an authentication check. Furthermore, we analysed how input validation is deployed in popular PHP frameworks.





\todo[inline]{Outline with section numbers (explain the structure of the paper)}


\subsection{Types of Analyzers}
There are different types of analyzers which all have a different scope. Most of them are specialized to a specific programming language, but some are also capable of analyzing multiple languages. An example for a multi-language analyzer is \href{http://pmd.sourceforge.net/pmd-4.3.0/cpd.html}{CPD} (Copy/Paste Detector) which is supposed to find duplicate code. It works with Java, JSP, C, C++, Fortran and PHP code.


Furthermore it is important to distinguish when to apply analyzers in the Software Development Lifecycle.
Most analyzer exist as stand-alone version and allow the integration in common development tools, like IDE's and CI tools.
Another promising approach is the integration in Software configuration systems (SCM) or in external code quality management tools, which enable professional issue management activities.

%TODO Definition for ci

Advantages and disadvantages of Continuous Integration (CI)\cite{SecurityinCI}:
\begin{itemize}
	\item[+] \textit{Immediate Notification}
	CI ensures that ongoing changes to the source code do not break the intent or design of the software. If a change does break the software, that break is identified immediately and can be fixed with a minimal cost and impact to the projects schedule.
	\item[+] \textit{Automated Testing and Deploying}
	CI enables many automation possibilities. The most useful automation area is testing in form of Unit- and Integration-Testing, to find problems after component integration and change introduced bugs in previously working components. 
	Finally a correct configured CI system can automate the deployment of software releases. 
	\item[+] \textit{Secure Development}
	By integrating security testing and secure code analysis, CI can be further leveraged to include secure development practices while minimizing the amount of extra effort required to get the benefits of secure development. Since it is tied to CI,
	security testing and secure code review begins when a project begins and runs continuously
	throughout project development. With CI, security vulnerabilities testing becomes part of the regression test bed, executed automatically with each successive build on the CI platform.

	\item[+] \textit{Changing Testing Economics}
	Using CI for build, test, and analysis automation has increased the depth and breadth of tests while also making them faster and less expensive. By making it cheap and easy to perform tests, teams are encouraged to test more and test sooner in the development cycle, reducing the cost of fixing bugs.
	
	\item[+] \textit{Trend and History}
	CI enables a higher management layer to view the history and trend of issues and builds.
	
	\item[-] \textit{CI Configuration}
	The configuration of a CI instance can be very troublesome and involves the understanding of many different tools. To create a working tool chain of testing, analyzing and building, with many thresholds and parameters the developer team has to understand every tool and have to tune parameters after gaining more experience.
\end{itemize}

%TODO definition code quality management and advantages 


\section{Input validation in popular frameworks}
\label{sec:input_validation}
\todo{Move this section to the end? - Brian}
Check those \url{http://codegeekz.com/20-best-php-frameworks-developers-august-2014/} and make a table how input validation is handled there...
The most common security weakness in any application is the failure to properly validate input from the environment. This weakness further leads to almost all of the major vulnerabilities in applications, including buffer overflow, SQL injection and a whole lot more. So, the maximum essential cautious measure that developers can take is to comprehensively authenticate the input that a software obtains. Certainly programs need to accept input, and computing a decent result depends on having a good input. There is a misconception that input can be trusted just because it is coming from some so-called trusted source. Input must not enter into the system without passing through various security methods.



\section{Static code analysis in IDE}
\label{sec:static_code_analysis_ide}
Static code analysis on IDE level is a common choice for many projects.
It is easy to integrate common analyzers like PMD and FindBugs in IDE's like like Eclipse, Netbeans or IntelliJ.
Advantages of integrating static code analysis in IDE's:
\begin{itemize}
	\item Live feedback during programming
	\item Easy mapping of issues to code
	\item possibility to fix issues early and instantly
	\item interactive education for developers
\end{itemize}

However disadvantages arise with bigger projects. 
Analyzers, especially data flow analyzers, tend to scale for big projects not very well. 
There is no central way to configure the analyzer rules, to improve the results and performance.
Many developer tend to suppress warnings from such analyzers, since they produce a lot of false positives.
Hence, it is desired to have a central and independent analyzer, which can run on the remote repository code.


As example IDE we used the popular OpenSource IDE Eclipse\footnote{\href{https://eclipse.org/}{https://eclipse.org/}}.
Beside its popularity we decided for Eclipse, because it is platform independent and highly extensible.
We could integrate our test analyzers PMD and FindBugs as plugins.




\section{Static code analysis in CI}
\label{sec:static_code_analysis_ci}
Continuous integration tools run independently Unit Tests and other jobs over remote repository code.
Hence, they should be the perfect place to apply static code analysis.

\subsection{Jenkins}
\label{subsec:jenkins}
Jenkins is a widely used tool to control and manage continuous integration tasks. Its main purpose is to monitor the execution of repeated jobs and present their outcomes.\footnote{From Jenkins Website, \href{https://wiki.jenkins-ci.org/display/JENKINS/Meet+Jenkins}{Meet Jenkins}}
We decided for Jenkins because it is OpenSource, highly extensible and the most popular CI tool.
Beside PMD and FindBugs, there are many more static analyzer available in the plugin repository.
In our research we found, almost every analyzer has a Jenkins plugin available.
Many OpenSource analyzers, like BrakeMan\footnote{\href{http://brakemanscanner.org/}{http://brakemanscanner.org/}}, Cppcheck\footnote{\href{http://cppcheck.sourceforge.net/}{http://cppcheck.sourceforge.net/}} as well as popular commercial tools like Coverity\footnote{\href{http://www.coverity.com/}{http://www.coverity.com/}} and Fortify\footnote{\href{http://www8.hp.com/de/de/software-solutions/application-security/}{http://www8.hp.com/de/de/software-solutions/application-security/}}.
But not all plugins provide a full analyzer.
Especially plugins for commercial tools like Coverity just provide a link to a corresponding web platform for code quality and issue management. 


\begin{figure}[!t]
	\centering
	\includegraphics[width=1\linewidth]{img/jenkins-code-analysis-plugins.png}
	\caption{Just a few used Jenkins plugins for static code analysis}
	\label{fig:jenkins-plugins}
\end{figure}


\subsection{Teamcity}
\label{subsec:static_code_analysis_teamcity}
Teamcity is like Jenkins a web application for continuous integration, from the company Jetbrains\footnote{\href{https://www.jetbrains.com/teamcity/}{https://www.jetbrains.com/teamcity/}}. In contrast to Jenkins, it is not open source, but free available with a limitation of 20 build configurations.
Also it claims to be easier to use and to configure than Jenkins.
It provides possibilities to run analyzers before or after the build process and to inspect resulting reports.
Furthermore it works together with static code analyzers in the commercial IntelliJ\footnote{\href{https://www.jetbrains.com/idea/}{https://www.jetbrains.com/idea/}} IDE from Jetbrains.

\begin{itemize}
	\item Installation: The Teamcity installation was a bit easier than Sonarqube because the database connection was configured in a wizard and similar it consists out of a webserver with webinterface (localhost:8111) and a database.
	
	\item Language support: All (since it is a highly customizable CI tool)
	
	\item Rule Management: The analyzers have to run externally and Teamcity will import the analysis results.
	Hence, it is much effort to configure and install different analyzers with different rulesets.
	
	\item Issue-Presentation: In the build details you can show issues in the Code Inspection tab, what present just the xml tree of the parsed reports, no filter options, no resolve options, no severity levels.
	
	\item Positive: Better user interface than Jenkins. With installed IDE plugin, you can jump to issue source-code directly in the IDE. 
	
	\item Negative: Hard to configure the external pmd and findbugs reports. Visualization of issues very limited. Two different inspections can not be processed during one build (skipped PMD report)
	
\end{itemize}





\section{Static code analysis in code quality management}
\label{sec:static_code_analysis_code_quality_management}

\subsection{SonarQube}
\label{subsec:sonarqube}
SonarQube\footnote{\href{http://www.sonarqube.org/}{http://www.sonarqube.org/}} is an OpenSource project, implemented as web application within its own webserver, with the aim to analyze and manage the quality of source code. 
Besides analysis against common coding guidelines, like duplicate code, missing comments and potential bugs, it also checks for several security issues (e.g. from the OWSASP Top 10 list).
Therefore it provides a central place to manage intuitively analysis rules from different analyzer extensions.
The main difference to CI tools, is the feature to manage the found issues.
Over plugins it is possible to extend the analysis scope to over 20 programming languages.


\begin{itemize}
\item Installation: Sonarqube consists out of 3 parts: the local webserver (localhost:9000), a database to load and store analysis results and the sonarqube runner, which analyses the code specified in a project property file.

\item Language support: Java and other languages available as plugins

\item Rule Management: Very intuitive and easy to configure rules for so called quality profiles. Very interesting, that you can manage the rules of all analyzer plugins in one menu. It comes already with default sonarqube rules with different severity levels, detailed descriptions in different categories. Also including security relevant rules e.g. from OWASPTop10 and CWEs. One can compose a static analysis out of those rules including rules from plugins like FindBugs and PMD.

\item Issue-Presentation: Dashboard as starting point presents overview very well with different informative metrics, mostly for code quality and presents the number of issues of the different severity levels.
In the menu issues and issue-drilldown one can sort the issue list, search for issues matching to specific rules and get more information to the issue and most important where the issue is located in the code.

\item Positive: Good overview over issues, fast analysis, good rule management, good issue management with assigning issues to users etc.

\item Negative: In Rule-Management one can filter rules by tags and categories (bugs, security bugs), but you can not use those filters in the issue management. This leads to a big list of issues where you cant distinguish e.g. between a unused code and security issue.
 
\end{itemize}

\subsection{Codeclimate}
\label{subsec:codeclimate}
\todo[inline]{Maybe we leave codeclimate out....its commercial...and would maybe be too much work to include it in our evaluation...to make it round we would have to include then also another IDE...too much work?}











We subdivide our evaluations by programming language because the analyzers and their main purpose differs heavily in respect to their target language.

\section{C/C++}

\subsection{Cppcheck}
Cppcheck is a very powerful analyzer for C and C++ which can find a lot different code issues, for instance out-of-bounds accesses, memory leaks or warn if obsolete or unsafe functions are used. Its main goal is to avoid false positives.

Cppcheck is easy to use and can be integrated in an continuous life cycle with very little effort.
Listing \ref{lst:cppcheck} shows how to run a Cppcheck analysis on a C/C++ project. When Cppcheck is started using this command, it will run with 4 threads (\texttt{-j 4}), performs all analysis it supports (\texttt{--enable=all}) and writes the results as xml file on disk after it examined any C/C++ file in the current folder and any folder below that (recursively).

\begin{lstlisting}[caption={Bash command to run Cppcheck},label={lst:cppcheck}]
$ cppcheck -j 4 --enable=all \
	--xml --xml-version=2 \
	. 2> cppcheck.xml
\end{lstlisting}


\subsubsection{Cppcheck \& Jenkins}
There is a dedicated \href{https://wiki.jenkins-ci.org/display/JENKINS/Cppcheck+Plugin}{Jenkins plugin for Cppcheck}. This plugin does not include the Cppcheck analysis nor does it issue a check. A run has to be issued as a build step which produces a XML-file containing the analysis results. The plugin will then, as a post-build task, read the XML-file and display the contents in a processed manner.

\subsubsection{Cppcheck \& SonarQube}
There are two plugins to integrate C/C++ static code analysis into SonarQube. There is a professional plugin released by SonarSource which is the company behind SonarQube. This plugin includes some analysis and coding rules but can also display the results of third party analyzers like Cppcheck. Since you can only use this plugin if you have a valid license, we could not try it on our own but there is a public demonstration site showing the analysis results for \href{http://nemo.sonarqube.org/dashboard/index/clang}{clang} and \href{http://nemo.sonarqube.org/dashboard/index/mysql}{MySQL}. 

The second plugin is sonar-cxx which is a community plugin, hence it is open source and free to use.

\subsection{RATS}
RATS (Rough Auditing Tool for Security) is a static code analyzer specialized to security related programming errors.

\begin{lstlisting}[caption={Bash command to run RATS},label={lst:rats}]
$ rats -w 3 --xml . > report.xml
\end{lstlisting}

\subsubsection{RATS \& Jenkins}
Unfortunately, we did not find a plugin to display RATS results with Jenkins.

\subsubsection{RATS \& SonarQube}






\section{Evaluation}
\label{sec:evaluation}
We decided to make a qualitative evaluation, with usability inspection heuristics described in the book Information Visualization by Kerren et al. \cite{InformationVisualizationBook}.
From the work of Hollingsed et al. on 15 years of usability inspection evaluation \cite{15yearsUsabilityEvaluation}, we derived the best method would be a combination of a cognitive walk-through, combined with the usability heuristic evaluation defined by Nielsen \cite{Nielsen:UsabilityInspectionMethods}.
This wide-used, informal, very cost efficient and effective method is proven to find with a appropriate skilled evaluator team 55 - 90\% of all usability problems.


The usability inspection over heuristic evaluation method uses a small group of usability experts, who evaluate a
user interface using a set of guidelines and noting the severity of
each usability problem and where it exists. 
We combine it with a cognitive walk-though, in the way, that three experts go through the tools with a cognitive expected path in context of applying static code analysis with an additional usability guidelines list for every stage.


We identified four stages of our walk-through and evaluate several usability guidelines in each stage:

\begin{itemize}
	\item Prepare analysis
	\begin{itemize}
		\item Is the tool easy and intuitive to configure?
		\item Is it possible to add external analyzers?
		\item Is it possible to configure the analyzers?
		\item Is it possible to view the rules?
		\item Is it possible to choose, add, edit, delete rules?
	\end{itemize}
	\item Run analysis
	\begin{itemize}
		\item Is it easy and intuitive to start the analysis?
		\item Is it possible to following the analysis progress?
		\item Is it possible to 
	\end{itemize}
	\item Evaluate analysis results
	\begin{itemize}
		\item Is there an issue overview with severity levels?
		\item Is it possible to view an issue trend/history?
		\item Is there a mapping from issue to source code position?
		\item Is there a description for every issue?
		\item Is the issue description easy to understand with solution suggestions?
		\item Is there a possibility to filter issues? (severity, category, tag, ...)
	\end{itemize}
	\item Manage analysis results
	\begin{itemize}
		\item Is it possible to assign issues to developers?
		\item Is is possible to edit issue status? (Resolved, False positive, ...)
	\end{itemize}
\end{itemize}




\subsection{Eclipse}
\label{subsec:evaluation_eclipse}

\subsection{Jenkins}
\label{subsec:evaluation_jenkins}

\subsection{Teamcity}
\label{subsec:evaluation_teamcity}

\subsection{SonarQube}
\label{subsec:evaluation_sonarqube}

\subsection{Codeclimate}
\label{subsec:evaluation_codeclimate}


\subsection{Comparation}
\label{subsec:comparation}



\section{Conclusion}
\label{sec:conclusion}
\begin{itemize}
	\item Conclusion about how input validation is done in frameworks, what can be better ...
	\item Conclusion, is it better to integrate static analysis in Jenkins or just use SonarQube
	....its really not that easy to find the right static code analyzer for your project with a specific programming language. There are lots of open source tools, but very old and just supported by Jenkins over hacks.
	\item Conclusion, is reporting in Jenkins useable? No...just in combination with code quality management tools
	\item Future work: Using many tools is basically a good idea, because more tools find potentially more vulnerabilities. A future approach would be to implement a tool that can filter all the generated reports. Thereby duplicate vulnerabilities findings can be merged and false positives can be reduced.
\end{itemize}






\section{Sources and useful links}
\begin{itemize}
	\item \href{http://www.crosstalkonline.org/storage/issue-archives/2010/201003/201003-Stiehm.pdf}{Building Security In Using Continuous Integration}
	\item \href{http://www.jetbrains.com/teamcity}{TeamCity}\\
	Supports static code analysis. Also, it can import analysis reports produced by these tools: PMD, PMD/CPD, FindBugs, Checkstyle or JSLint.
	\item \href{http://jenkins-ci.org}{Jenkins}
	\item \href{http://continuum.apache.org}{Apache Continuum}	
	\item \href{http://codeclimate.com}{codeclimate.com}
\end{itemize}

\bibliographystyle{IEEEtran}
% argument is your BibTeX string definitions and bibliography database(s)
\bibliography{references}

\end{document}


