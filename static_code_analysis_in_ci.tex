%%
%% static_code_analysis_in_ci.tex
%% V0.1
%% 2015/01/22
%% by 
%% Sebastian Funke
%% Hamza Zulfiqar
%% Brian Pfretzschner
%% See:
%% https://github.com/hzulfiqar/SecSoftDev
%% for current contact information.
%%


\documentclass[conference]{IEEEtran}


% *** PACKAGES ***
%\usepackage{algorithmic}
%\usepackage{array}
%\usepackage{mdwmath}
%\usepackage{mdwtab}
%\usepackage{eqparbox}
%\usepackage{fixltx2e}
%\usepackage{stfloats}
\usepackage{cite}       % http://www.ctan.org/tex-archive/macros/latex/contrib/supported/cite/
\ifx\pdfoutput\undefined
\usepackage{graphicx}   % http://www.ctan.org/tex-archive/macros/latex/required/graphics/
\else
\usepackage[pdftex]{graphicx}
\fi
\usepackage{subfigure}  % http://www.ctan.org/tex-archive/macros/latex/contrib/supported/subfigure/
\usepackage{url}        % http://www.ctan.org/tex-archive/macros/latex/contrib/other/misc/
\usepackage[cmex10]{amsmath}    % http://www.ctan.org/tex-archive/macros/latex/required/amslatex/math/
%\usepackage{amsfonts}
\interdisplaylinepenalty=2500
\ifx\pdfoutput\undefined
\usepackage[hypertex]{hyperref}
\else                   % http://www.ctan.org/tex-archive/macros/latex/contrib/supported/hyperref/
\usepackage[pdftex,hypertexnames=false]{hyperref}
\fi
\usepackage[colorinlistoftodos,prependcaption,textsize=tiny]{todonotes}



% correct bad hyphenation here
\hyphenation{op-tical net-works semi-conduc-tor}


\begin{document}
%
% paper title
\title{Integration of Static Security Code Analysis\\in Continuous Integration Lifecycles}



%\author{\IEEEauthorblockN{Sebastian Funke}
%\IEEEauthorblockA{Secure Software Engineering\\
%TU Darmstadt\\
%sebastian.funke@stud.tu-darmstadt.de}
%\and
%\IEEEauthorblockN{Brian Pfretzschner}
%\IEEEauthorblockA{Secure Software Engineering\\
%TU Darmstadt\\
%brian.pfretzschner@stud.tu-darmstadt.de}
%\and
%\IEEEauthorblockN{Hamza Zulfiqar}
%\IEEEauthorblockA{Secure Software Engineering\\
%TU Darmstadt\\
%hamza.zulfiqar@stud.tu-darmstadt.de}}



\author{\authorblockA{Sebastian Funke, Brian Pfretzschner, Hamza Zulfiqar}
	\authorblockA{Center for Advanced Security Research Darmstadt\\
		Department of Computer Science\\
		Technische Universit\"at Darmstadt, Germany}}




% use for special paper notices
%\IEEEspecialpapernotice{(Invited Paper)}




% make the title area
\maketitle


\begin{abstract}
\boldmath
Static code analysis should run frequently in an continuous integration lifecycle. Each run produces lots of information that need to be reviewed, evaluated and integrated in the ongoing development process. Therefore, the analysis results should be reworked, concentrated and presented in a helpful manner.
\end{abstract}

% no keywords

% For peerreview papers, this IEEEtran command inserts a page break and
% creates the second title. It will be ignored for other modes.
\IEEEpeerreviewmaketitle



\section{Introduction}
% no \IEEEPARstart
\todo[inline]{Content from our slides...about input validation and how static code analysis works.
Limitation on open source, static analysis tools. Jenkins, SonarQube...and why\cite{IEEEexample:bluebookarticle}.
Content of our paper, what comes when blabla...}



We evaluate the vulnerability reporting capabilities in Jenkins and the open source quality management tool SonarQube.
We used the popular CI tool Jenkins on a NIST standardized C test project\footnote{\url{http://samate.nist.gov/SRD/testsuite.php}} with a variety of vulnerabilities.
Thereby we included a couple of static analysis tools in Jenkins for finding bugs and vulnerabilities during build and after the build process.
we address that what needs validation, how to perform input validation and how to respond when an input fails an authentication check. Furthermore, we analysed how input validation is deployed in popular PHP frameworks.

\subsection{Types of Analyzers}
There are different types of analyzers which all have a different scope. Most of them are specialized to a specific programming language, but some are also capable of analyzing multiple languages. An example for a multi-language analyzer is \href{http://pmd.sourceforge.net/pmd-4.3.0/cpd.html}{CPD} (Copy/Paste Detector) which is supposed to find duplicate code. It works with Java, JSP, C, C++, Fortran and PHP code.

\section{Input validation in popular frameworks}
\label{sec:input_validation}
Check those \url{http://codegeekz.com/20-best-php-frameworks-developers-august-2014/} and make a table how input validation is handled there...
The most common security weakness in any application is the failure to properly validate input from the environment. This weakness further leads to almost all of the major vulnerabilities in applications, including buffer overflow, SQL injection and a whole lot more. So, the maximum essential cautious measure that developers can take is to comprehensively authenticate the input that a software obtains. Certainly programs need to accept input, and computing a decent result depends on having a good input. There is a misconception that input can be trusted just because it is coming from some so-called trusted source. Input must not enter into the system without passing through various security methods.

\section{Static code analysis in Jenkins}
\label{sec:static_code_analysis_jenkins}
Used test suite: wireshark 1.8 from NIST testsuites with 85 vulnerabilities.
Because its C, because its one of the most security critical languages and there are many good analyzers for C.
Why not Juliet TestSuite with 65 000 vulnerabilities?
Because they are collections of testcases and not a easily build-able project and the analysis and build process would take to long to evaluate the reporting features of the analyzers.


Todo: Table with tools with pros and cons

\begin{figure}[!t]
	\centering
	\includegraphics[width=1\linewidth]{img/jenkins-code-analysis-plugins.png}
	\caption{Just a few used Jenkins plugins for static code analysis}
	\label{fig:jenkins-plugins}
\end{figure}


\section{Static code analysis in SonarQube}
\label{sec:static_code_analysis_sonarqube}

We subdivide our evaluations by programming language because the analyzers and their main purpose differs heavily in respect to their target language.

\section{C/C++}

\subsection{Cppcheck}
Cppcheck is a very powerful analyzer for C and C++ which can find a lot different code issues, for instance out-of-bounds accesses, memory leaks or warn if obsolete or unsafe functions are used. Its main goal is to avoid false positives.

Cppcheck is easy to use and can be integrated in an continuous life cycle with very little effort.
Listing \ref{lst:cppcheck} shows how to run a Cppcheck analysis on a C/C++ project. When Cppcheck is started using this command, it will run with 4 threads (\texttt{-j 4}), performs all analysis it supports (\texttt{--enable=all}) and writes the results as xml file on disk after it examined any C/C++ file in the current folder and any folder below that (recursively).

\begin{lstlisting}[caption={Bash command to run Cppcheck},label={lst:cppcheck}]
$ cppcheck -j 4 --enable=all \
	--xml --xml-version=2 \
	. 2> cppcheck.xml
\end{lstlisting}


\subsubsection{Cppcheck \& Jenkins}
There is a dedicated \href{https://wiki.jenkins-ci.org/display/JENKINS/Cppcheck+Plugin}{Jenkins plugin for Cppcheck}. This plugin does not include the Cppcheck analysis nor does it issue a check. A run has to be issued as a build step which produces a XML-file containing the analysis results. The plugin will then, as a post-build task, read the XML-file and display the contents in a processed manner.

\subsubsection{Cppcheck \& SonarQube}
There are two plugins to integrate C/C++ static code analysis into SonarQube. There is a professional plugin released by SonarSource which is the company behind SonarQube. This plugin includes some analysis and coding rules but can also display the results of third party analyzers like Cppcheck. Since you can only use this plugin if you have a valid license, we could not try it on our own but there is a public demonstration site showing the analysis results for \href{http://nemo.sonarqube.org/dashboard/index/clang}{clang} and \href{http://nemo.sonarqube.org/dashboard/index/mysql}{MySQL}. 

The second plugin is sonar-cxx which is a community plugin, hence it is open source and free to use.

\subsection{RATS}
RATS (Rough Auditing Tool for Security) is a static code analyzer specialized to security related programming errors.

\begin{lstlisting}[caption={Bash command to run RATS},label={lst:rats}]
$ rats -w 3 --xml . > report.xml
\end{lstlisting}

\subsubsection{RATS \& Jenkins}
Unfortunately, we did not find a plugin to display RATS results with Jenkins.

\subsubsection{RATS \& SonarQube}

\section{Evaluation of reporting capabilities}
\label{sec:evaluation}
Definition for userfriendly vulnerability reporting needed!
Metrics for evaluation of reports needed and need to be mapped on the useabillity definition. 

\subsection{Jenkins}
\label{sec:evaluation_jenkins}

\subsection{SonarQube}
\label{sec:evaluation_sonarqube}




\section{Conclusion}
\label{sec:conclusion}
\begin{itemize}
	\item Conclusion about how input validation is done in frameworks, what can be better ...
	\item Conclusion, is it better to integrate static analysis in Jenkins or just use SonarQube
	....its really not that easy to find the right static code analyzer for your project with a specific programming language. There are lots of open source tools, but very old and just supported by Jenkins over hacks.
	\item Conclusion, is reporting in Jenkins useable
	\item Future work: Using many tools is basically a good idea, because more tools find potentially more vulnerabilities. A future approach would be to implement a tool that can filter all the generated reports. Thereby duplicate vulnerabilities findings can be merged and false positives can be reduced.
\end{itemize}






\section{Sources and useful links}
\begin{itemize}
	\item \href{http://www.crosstalkonline.org/storage/issue-archives/2010/201003/201003-Stiehm.pdf}{Building Security In Using Continuous Integration}
	\item \href{http://www.jetbrains.com/teamcity}{TeamCity}\\
	Supports static code analysis. Also, it can import analysis reports produced by these tools: PMD, PMD/CPD, FindBugs, Checkstyle or JSLint.
	\item \href{http://jenkins-ci.org}{Jenkins}
	\item \href{http://continuum.apache.org}{Apache Continuum}	
	\item \href{http://codeclimate.com}{codeclimate.com}
\end{itemize}




\bibliographystyle{IEEEtran}
% argument is your BibTeX string definitions and bibliography database(s)
\bibliography{./references}


\end{document}


